%% Copyright 2019-2020 Elsevier Ltd
%% 
%% This file is part of the 'CAS Bundle'.
%% --------------------------------------
%% 
%% It may be distributed under the conditions of the LaTeX Project Public
%% License, either version 1.2 of this license or (at your option) any
%% later version.  The latest version of this license is in
%%    http://www.latex-project.org/lppl.txt
%% and version 1.2 or later is part of all distributions of LaTeX
%% version 1999/12/01 or later.
%% 
%% The list of all files belonging to the 'CAS Bundle' is
%% given in the file `manifest.txt'.
%% 
%% Template article for cas-sc documentclass for 
%% double column output.

%\documentclass[a4paper,fleqn,longmktitle]{cas-sc}
\documentclass[a4paper]{cas-sc}
% \usepackage[numbers]{natbib}
\usepackage[round,sort&compress]{natbib}
\usepackage{amsmath}
\usepackage{amssymb}
\usepackage{booktabs}
% \usepackage{stfloats}
\usepackage[capitalize]{cleveref}
\usepackage{graphicx}
\usepackage{subfig}
\usepackage{enumerate} 
\usepackage{setspace}  
\usepackage{caption}
\usepackage{longtable} 
\usepackage{xcolor} 
\usepackage{tikz} 
\usepackage{tabularx}
\usepackage{enumitem}
\usepackage{bm}
\usepackage{float}
\usepackage{lineno}
\usepackage{subfig}
\usepackage{subcaption}
\usepackage{overpic}
\usepackage{soul}
\usepackage{array}
\usepackage{changepage} % 提供 adjustwidth 环境
\usepackage{ragged2e} % for \justifying
\newcommand{\lightblue}{rgb:red,0.8;green,1;blue,1.2}
\definecolor{mygray}{rgb}{1,1,1} %
\sethlcolor{mygray} % 
\newcommand{\adj}[1]{\raisebox{-2pt}[\height][\depth]{#1}}
% Uncomment and use as if needed
%\newtheorem{theorem}{Theorem}
%\newtheorem{lemma}[theorem]{Lemma}
%\newdefinition{rmk}{Remark}
%\newproof{pf}{Proof}
%\newproof{pot}{Proof of Theorem \ref{thm}}
\setlength{\parskip}{0.2em}


% 定义柔和的蓝色
\definecolor{softblue}{RGB}{0, 102, 204}

%\newcommand{\revisedwd}[1]{{\color{softblue}\hypersetup{citecolor=softblue}
%\hypersetup{linkcolor=softblue}#1}} %\color{softblue}

\newcommand{\revisedwd}[1]{{#1}} %\color{softblue}

\captionsetup{
    justification=justified,%
}

\newenvironment{tttabular}[1]%
{\ttfamily \begin{tabular}{#1}}%
{\end{tabular}}

\begin{document}
\linenumbers
\let\WriteBookmarks\relax
\def\floatpagepagefraction{1}

% Short title
\shorttitle{}

% Main title of the paper
\title [mode = title]{Automatic 3D Railway Track Extraction and Reconstruction with Kalman Filter in Multiple Aerial Images}                      
% Title footnote mark
% eg: \tnotemark[1]


% First author
%
% Options: Use if required
% eg: \author[1,3]{Author Name}[type=editor,
%       style=chinese,
%       auid=000,
%       bioid=1, 
%       prefix=Sir,
%       orcid=0000-0000-0000-0000,
%       facebook=<facebook id>,
%       twitter=<twitter id>,
%       linkedin=<linkedin id>,
%       gplus=<gplus id>]
\author[1]{Dong Wei}
\fnmark[1]
% Corresponding author indication
%\cormark[1]
% Footnote of the first author
%\fnmark[1]
% Email id of the first author
\ead{weidong@whu.edu.cn}
% Second author
\author[1]{Xiaotong Li}
\fnmark[1]
\author[1]{Yongjun Zhang}
\cormark[1]
\ead{zhangyj@whu.edu.cn}
% Third author
\author[2]{Chang Li}
\ead{lichang@ccnu.edu.cn}

\author[1]{Ziqian Huang}
\fnmark[1]


% Address/affiliation
\affiliation[1]{organization={School of Remote Sensing and Information Engineering, Wuhan University},
    %addressline={Bayi Road }, 
    city={Wuhan},
    % citysep={}, % Uncomment if no comma needed between city and postcode
    postcode={430072}, 
    % state={},
    country={P.R.China}}



% Address/affiliation
\affiliation[3]{organization={College of Urban and Environmental Science, Central China Normal University},
    %addressline={Bayi Road }, 
    city={Wuhan},
    % citysep={}, % Uncomment if no comma needed between city and postcode
    postcode={430072}, 
    % state={},
    country={P.R.China}}

% Corresponding author text
\cortext[cor1]{Corresponding author}
\fntext[1]{Co-first authors. 
}

% Here goes the abstract
\begin{abstract}
Three-dimensional (3D) lines require further enhancement in both clustering and triangulation. 
Line clustering assigns multiple image lines to a single 3D line to eliminate redundant 3D lines.
Currently, it depends on the fixed and empirical parameter.
However,
a loose parameter could lead to over-clustering, 
while a strict one may cause redundant 3D lines. 
Due to the absence of the ground truth, 
the assessment of line clustering remains unexplored.
Additionally, 
3D line triangulation, 
which determines the 3D line segment in object space, 
is prone to failure due to its sensitivity to positional and camera errors.

\noindent This paper aims to improve the clustering and triangulation of 3D lines and to offer a reliable evaluation method. 
(1) To achieve accurate clustering, 
we introduce a probability model,
which uses the prior error of the structure from the motion,
to determine adaptive thresholds;
\end{abstract}

% Use if graphical abstract is present
% \begin{graphicalabstract}
% \includegraphics{figs/grabs.pdf}
% \end{graphicalabstract}

% Research highlights
%\begin{highlights}
%\item Research highlights item 1
%\item Research highlights item 2
%\item Research highlights item 3
%\end{highlights}

% Keywords
% Each keyword is seperated by \sep
\begin{keywords}
    3D line segments
    \sep line clustering
    \sep line triangulation 
    \sep 3D line evaluation
\end{keywords}

\maketitle

\input{sec/Introduction.tex}
\section{Related works}

Railway detection and line reconstruction are essential components of accurately modeling railway infrastructure. This study reviews three key aspects: point cloud-based methods for railway detection, image-based approaches for enhanced recognition, and 3D line reconstruction techniques for precise segment modeling.

\subsection{Point Cloud-Based railway detection}
3D point clouds, owing to their efficiency and precision, are widely used for railway segment detection, with both traditional and modern methods contributing to advancements in the field.
Traditional approaches primarily rely on data fusion and geometric models. \cite{Beger2011} demonstrated that fusing aerial imagery with LiDAR data significantly enhances railway centerline reconstruction, highlighting the benefits of multi-source data integration. Similarly, \cite{Yang2014} utilized shape and intensity features to model railway tracks, while \cite{SanchezRodriguez2019} adopted a coarse-to-fine RANSAC strategy to refine railway tunnel power line detection. These methods showcase the potential of tailored geometric models in specific applications but suffer from sensitivity to data quality and acquisition conditions.
Infrastructure-focused approaches, such as those by \cite{Cserep2022}, emphasize detailed railway fragmentation and infrastructure recognition using dense LiDAR data, enabling detailed mapping of complex structures. 
\cite{Ye2022} extended this to high-definition lane extraction in curved road scenarios, underlining the adaptability of point cloud data for intricate transportation features. 
\cite{Ariyachandra2023} further enhanced railway topology analysis by integrating geometric constraints, paving the way for more reliable detection and modeling of railway components, including catenary structures \citep{Zhang2023a}.

Although these methods provide valuable information, they rely heavily on manually defined features, which limits their scalability across diverse environments. 
To address these challenges, deep learning approaches have emerged as a promissing solution.
\cite{Ma2022} introduced BoundaryNet, leveraging MLS point clouds and satellite imagery to extract road boundaries with high accuracy. 
\cite{Zhang2022} tackle complex geometries, such as overhead wires, by learning intricate patterns in airborne LiDAR data. 
Simulation frameworks like TrainSim \citep{DAmico2023} provide a synthetic system to train deep learning models, enhancing their robustness in real-world applications.
Real-time detection systems, such as the fusion-based approach \citep{Tang2024}, combine LiDAR and camera data to improve efficiency and accuracy, and the anomaly detection techniques by \cite{Ge2024} integrate semi-supervised learning and decision fusion for robust railway inspection.
Despite these advances, deep learning methods demand substantial computational resources and rely on extensive annotated datasets, as exemplified by WHU-Railway3D \citep{Qiu2024}, which serves as a benchmark for railway semantic segmentation. 

\subsection{Image-Based railway detection}
Image-based railway detection has been revolutionized by deep learning, leveraging recent advances in pattern recognition to improve performance across diverse scenarios. Unlike point cloud-based methods, image-based techniques benefit from the ubiquity of cameras and the rich contextual information in images, making them ideal for railway recognition.
Early approaches, such as those by \cite{Yang2022b}, combined discretization, filtering, and reconstruction to detect railway tracks in UAV images, demonstrating their applicability for inspection purposes.
\cite{Zheng2022} improved feature integration through cross-layer refinement, enabling robust lane detection under varying environmental conditions. These methods laid the foundation for incorporating deeper geometric modeling and advanced architectures.
Recent methods have focused on improving the representation of complex structures.
For instance, \cite{Chae2023} reconstructed lane lines as polylines, emphasizing the importance of geometric modeling for autonomous driving. Transformer-based methods \citep{Chen2023,Luo2023,Yao2023} introduced a paradigm shift by capturing long-range dependencies, enabling accurate detection even in complex or sparse data scenarios. Flexible anchoring mechanisms like FLAMNet \citep{Ran2023} adapt to challenges such as lane curvature and occlusions, addressing limitations of fixed-detection models.

Semantic segmentation remains a cornerstone for image-based railway detection.
\cite{Weng2023} enhanced traditional architectures like DeepLabV3+, providing precise track extraction for automated maintenance. Furthermore,
\cite{zhang2024enhanced} advanced curved lane detection with ECPNet, a method that combines global spatial understanding with local precision, showcasing the role of adaptive techniques in refining detection accuracy.
Geometry-driven methods, such as the topology-guided approach by \cite{Yang2022a}, bridge the gap between traditional and deep learning methods by leveraging structural constraints to enhance detection robustness. 
This highlights the enduring relevance of geometric insights in complementing data-driven techniques.

\subsection{3D line reconstruction}
The forward intersection of multi-view line segments is performed based on segment association, where multiple planes constructed by the imaging center and image segments are intersected to solve for accurate 3D line segments. The overall framework can be broadly categorized into two types:
Strict association results: These methods directly fit line segments in 3D space under strict threshold constraints \citep{Jain2010}, or iteratively adjust and optimize 3D line segments through more rigorous back-projection models \citep{Schmid1997, HOFER, LiuCVPR}.
Reconstruction-based methods: These approaches reconstruct 3D line segments from matched two-view line segments and then select the optimal representative line segment based on the geometric consistency of the segments \citep{WEI2022}. Although these methods typically achieve suboptimal positioning accuracy compared to the first type, they are less sensitive to gross errors in matching and feature association, balancing reconstruction efficiency and quality, and are suitable for scenarios with more relaxed matching and association constraints.
The forward intersection of multi-view line segments is more complex than that of point features. When line segments are close to the epipolar plane, significant intersection errors may occur, resulting in inaccurate segment positioning \citep{Hartley2003}. Although various solutions have been proposed to address this issue, their fundamental idea remains consistent: constructing an optimization function by integrating geometric information and back-projection errors, rather than solely focusing on the error model of the line segments themselves. Among these, coplanarity between line segments serves as an effective geometric constraint. For distant adjacent line segments, 3D line segments are solved jointly based on coplanarity constraints \citep{OK}. Other studies reconstruct 3D line segments, group them into planes through fitting, and project the grouped segments onto the fitted planes to obtain 3D line segments.

Recently, deep learning methods have been explored for plane detection and grouping based on the orientation of 3D lines, using plane associations for forward intersection \citep{wang2020reconstruction}. This method produces visually regularized line clouds, but its actual positioning accuracy is significantly affected by coplanarity thresholds and scene structure. In the latest studies \citep{LiuCVPR}, two approaches were proposed to optimize the forward intersection of line segments:
Vanishing point-based intersection constraints: This approach corrects segment errors using vanishing point constraints. It remains based on the coplanarity of line segments but allows the use of existing vanishing point detection algorithms for segment grouping.
Deep learning-based matching of homologous points on line segments: Since the intersection of homologous points is only invalid near the epipolar line, this method provides robust positional constraints.
These advancements demonstrate promising directions for improving both the accuracy and robustness of line segment forward intersection.
\section{Methodology}

The flow of our methodology is presented in Figure 1.
We take aerial images of the railway area as input,
for which SfM and point clouds are used in advanced with existing software.
We first extract the image line and convert it to a space line with the locally optimal plane of the point cloud.
Then,
we cluster the single 3D line to RT candidates with the frame work derived from DBSCAN,  
during which the texture information of multiple images extracted from ResNet is used as one of the inlier distance.
Having obtained the RT cluster, 
we then trace and reconstruct the vector-based RT in the Kalman framwork,
which fully exploits the RT structure and the multi-view geometries to resolve the uncertainty caused by initial image line segment extraction and the point cloud error.
The railway-track pair is the start seed of our Kalman method.
We first convert image lines to 3D lines with the local optimal plane of the point cloud.
Then,
we cluster the single 3D line to RT pairs with DBSCAN frame work,  
during which the deep feature of multiple images is used as the inlier distance.





\subsection{Railway track with Kalman filter}

Let us first introduce the Kalman filter that optimize the \textit{RLP} without the geometry constraint between each other.
The state to be estimated is the two points and directions on the the \textit{RLP}:
\begin{equation}
\mathbf{x} = \begin{bmatrix}
    \mathbf p_1,\mathbf d_1,\mathbf p_2,\mathbf d_2 
\end{bmatrix}^ \top,
\end{equation}
where $\mathbf p_i=\left(x_i, y_i, z_i\right)$ denotes the position 
and $\mathbf d_i=\left(dx_i, dy_i, dz_i\right)$ is the normalized vector represents the direction.
Then,
the linear stochastic difference equation govern the state transition is
\begin{equation}
        \mathbf{x}_{k+1}= 
        \operatorname{diag} \left(\mathrm I,t \! \cdot \! \mathrm I, \mathrm I, t \! \cdot \! \mathrm I \right) \mathbf{x}_{k} + \mathbf{w}_k,
        \label {eq_statetransition}
\end{equation}
where $\mathrm I_{3\times3}$ is the identity matrix and $p \left(\mathbf w_k \right) \sim N(0, Q)$.
In each transition,
we can reconstruct the 3D line with multiple images (\cref*{sec_linereconstruction} ),
with which the observed position and direction related to $\mathbf x_k$ can be obtained.
Thus,
the measurement can be the same as $\mathbf x_k$,
and the general Kalman filter to track the \textit{RPL} is
\begin{equation}
    \mathbf {\hat x}_k =\mathbf {\hat x}_k^{\mbox -}+ \mathrm K\left(\mathbf z_k - \mathbf {\hat x}_k^{\mbox -}\right),\quad
    \mathbf z_k = \mathbf x_k+ \mathbf v_k,
\end{equation}
where $p \left(\mathbf v_k \right) \sim N(0, R)$,
$\mathbf {\hat x}_k^{\mbox -}$ is the prediction with \cref{eq_statetransition},
and $\mathrm K$ is the Kalman gain that iteratively calculated from the estimate error.
Please refer to (ref) for the details of Kalman gain. 

Now we add two geometry constraints of the \textit{RLP} to the filter process:
(1) $\mathbf d_1$ and $\mathbf d_2$ should be as close as possible;
and (2) The change in distance between $\mathbf p_1$ and $\mathbf p_2$ is as small as possible.


\subsection{Accurate railway line reconstruction}
\label{sec_linereconstruction}

\subsection{The seed generation for railway track}
We construct the rough 3D line based on the dense point cloud.
Given the end point $\mathbf p$ of a 2D line segment,
we randomly sample three space points around $\mathbf p$ to construct the plane $\pi$,
and we cast a ray $\mathbf r$ passing through $\mathbf p$ from the camera center.
Then,
the 3D point candidate $\mathrm P \in R ^ {3\times1}$ for $\mathbf p$ can be obtained by $\mathbf r$-to-$\pi$ intersection,
and the candidate 3D line $\mathrm L$ can be represented by the two 3D point:
\begin{equation}
    L =\left\{\textit{inter} \left(\pi,\mathbf r_1\right),\textit{inter} \left(\pi,\mathbf r_2\right)  \right\},
\end{equation}
where $\mathbf r_1$ and $\mathbf r_2$ are the rays of the two endpoints,
and \textit{inter} calculates the ray-to-plane intersection.
After random sampling of $n$ times,
we obtain a set of candidate 3D lines $\left\{ L\right\}_{i=1}^n$ and use the LMEDS algorithm,
which does not require an inlier threshold, to confirm the best 3D line for a 2D line:
\begin{equation}
    L^* = \arg\min_{L_i} \text{median} \left\{ d_{ij}\right\}_{j=1}^n ,
\end{equation}
where $d_{ij}$ is the projection distance between $L_i$ and $L_j$.  

We group two 3D lines as a RT pair based on their angle $\theta_{i,j}$,
overlap $o_{i,j}$,
and projection distance $d_{i,j}$:
\begin{equation}
   \left\{ RT= \left(L_i, L_j\right) \mid \theta_{i,j} < t_\theta, o_{i,j} > t_o, d_{i,j} \in I  \right\},
    \label{eq_geometrycons}
\end{equation}
$\theta_{i,j}$ and $o_{i,j}$are easy to choose,
e.g.,
$5^\circ$ and 60\%,
because the RT pair is parallel and highly overlapped;
while the interval $I$ needs the rough width $\omega$ between the two RT,
which can be acquired from construction standards or point clouds.
We recommend setting $I=\left[2/3\omega,4/3\omega\right]$ that uses one-third of $\omega$ as the margin of error.
Because a 3D line may satisfy \cref{eq_geometrycons} with many others,
the greedy algorithm is used to assign the candidate pair,
which uses the sum of the overlap rate as the maximum score.

We sort the RT based on their scores of the geometry alignment and select the top 10\% RT and use contextual information to further validate the RT pair.
In detail,
if the RT's central line is within $1^\circ$ and $t$ projection distance with another RT,
its score is increased by $\mathcal{N}\left(\mu, \left(t/3\right)^2\right)$.
we use the global average pooling layer in ResNet50 to describe the feature of the RT pair.
Because it has been trained on massive amounts of data and can capture texture information for classification in the absence of labels.
Also, 
we re-transform the image blocks to reduce the ambiguity caused by scale and rotation.
After extraction of RT features, 
we use DBSCAN to group them with the cosine distance,
and retain the group with the highest number as the seeds of RT.














\bibliographystyle{plainnat}
\bibliography{cas-refs}


\end{document}

