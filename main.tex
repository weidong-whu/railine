%% Copyright 2019-2020 Elsevier Ltd
%% 
%% This file is part of the 'CAS Bundle'.
%% --------------------------------------
%% 
%% It may be distributed under the conditions of the LaTeX Project Public
%% License, either version 1.2 of this license or (at your option) any
%% later version.  The latest version of this license is in
%%    http://www.latex-project.org/lppl.txt
%% and version 1.2 or later is part of all distributions of LaTeX
%% version 1999/12/01 or later.
%% 
%% The list of all files belonging to the 'CAS Bundle' is
%% given in the file `manifest.txt'.
%% 
%% Template article for cas-sc documentclass for 
%% double column output.

%\documentclass[a4paper,fleqn,longmktitle]{cas-sc}
\documentclass[a4paper]{cas-sc}
% \usepackage[numbers]{natbib}
\usepackage[round,sort&compress]{natbib}
\usepackage{amsmath}
\usepackage{amssymb}
\usepackage{booktabs}
% \usepackage{stfloats}
\usepackage[capitalize]{cleveref}
\usepackage{graphicx}
\usepackage{subfig}
\usepackage{enumerate} 
\usepackage{setspace}  
\usepackage{caption}
\usepackage{longtable} 
\usepackage{xcolor} 
\usepackage{tikz} 
\usepackage{tabularx}
\usepackage{enumitem}
\usepackage{bm}
\usepackage{float}
\usepackage{lineno}
\usepackage{subfig}
\usepackage{subcaption}
\usepackage{overpic}
\usepackage{soul}
\usepackage{array}
\usepackage{changepage} % 提供 adjustwidth 环境
\usepackage{ragged2e} % for \justifying
\newcommand{\lightblue}{rgb:red,0.8;green,1;blue,1.2}
\definecolor{mygray}{rgb}{1,1,1} %
\sethlcolor{mygray} % 
\newcommand{\adj}[1]{\raisebox{-2pt}[\height][\depth]{#1}}
% Uncomment and use as if needed
%\newtheorem{theorem}{Theorem}
%\newtheorem{lemma}[theorem]{Lemma}
%\newdefinition{rmk}{Remark}
%\newproof{pf}{Proof}
%\newproof{pot}{Proof of Theorem \ref{thm}}
\setlength{\parskip}{0.2em}


% 定义柔和的蓝色
\definecolor{softblue}{RGB}{0, 102, 204}

%\newcommand{\revisedwd}[1]{{\color{softblue}\hypersetup{citecolor=softblue}
%\hypersetup{linkcolor=softblue}#1}} %\color{softblue}

\newcommand{\revisedwd}[1]{{#1}} %\color{softblue}

\captionsetup{
    justification=justified,%
}

\newenvironment{tttabular}[1]%
{\ttfamily \begin{tabular}{#1}}%
{\end{tabular}}

\begin{document}
\linenumbers
\let\WriteBookmarks\relax
\def\floatpagepagefraction{1}

% Short title
\shorttitle{}

% Main title of the paper
\title [mode = title]{
Accurate 3D Railway Track Extraction from Aerial Images}                      
% Title footnote mark
% eg: \tnotemark[1]


% First author
%
% Options: Use if required
% eg: \author[1,3]{Author Name}[type=editor,
%       style=chinese,
%       auid=000,
%       bioid=1, 
%       prefix=Sir,
%       orcid=0000-0000-0000-0000,
%       facebook=<facebook id>,
%       twitter=<twitter id>,
%       linkedin=<linkedin id>,
%       gplus=<gplus id>]
\author[1]{Dong Wei}
\fnmark[1]
% Corresponding author indication
%\cormark[1]
% Footnote of the first author
%\fnmark[1]
% Email id of the first author
\ead{weidong@whu.edu.cn}
% Second author
\author[1]{Xiaotong Li}
\fnmark[1]
\author[1]{Yongjun Zhang}
\cormark[1]
\ead{zhangyj@whu.edu.cn}
% Third author
\author[2]{Chang Li}
\ead{lichang@ccnu.edu.cn}

\author[1]{Ziqian Huang}
\fnmark[1]


% Address/affiliation
\affiliation[1]{organization={School of Remote Sensing and Information Engineering, Wuhan University},
    %addressline={Bayi Road }, 
    city={Wuhan},
    % citysep={}, % Uncomment if no comma needed between city and postcode
    postcode={430072}, 
    % state={},
    country={P.R.China}}



% Address/affiliation
\affiliation[3]{organization={College of Urban and Environmental Science, Central China Normal University},
    %addressline={Bayi Road }, 
    city={Wuhan},
    % citysep={}, % Uncomment if no comma needed between city and postcode
    postcode={430072}, 
    % state={},
    country={P.R.China}}

% Corresponding author text
\cortext[cor1]{Corresponding author}
\fntext[1]{Co-first authors. 
}

% Here goes the abstract
\begin{abstract}
Three-dimensional (3D) lines require further enhancement in both clustering and triangulation. 
Line clustering assigns multiple image lines to a single 3D line to eliminate redundant 3D lines.
Currently, it depends on the fixed and empirical parameter.
However,
a loose parameter could lead to over-clustering, 
while a strict one may cause redundant 3D lines. 
Due to the absence of the ground truth, 
the assessment of line clustering remains unexplored.
Additionally, 
3D line triangulation, 
which determines the 3D line segment in object space, 
is prone to failure due to its sensitivity to positional and camera errors.

\noindent This paper aims to improve the clustering and triangulation of 3D lines and to offer a reliable evaluation method. 
(1) To achieve accurate clustering, 
we introduce a probability model,
which uses the prior error of the structure from the motion,
to determine adaptive thresholds;
\end{abstract}

% Use if graphical abstract is present
% \begin{graphicalabstract}
% \includegraphics{figs/grabs.pdf}
% \end{graphicalabstract}

% Research highlights
%\begin{highlights}
%\item Research highlights item 1
%\item Research highlights item 2
%\item Research highlights item 3
%\end{highlights}

% Keywords
% Each keyword is seperated by \sep
\begin{keywords}
    3D line segments
    \sep line clustering
    \sep line triangulation 
    \sep 3D line evaluation
\end{keywords}

\maketitle

\section{Introduction}

Currently,
the length of the railway has exceeded 1.3 million kilometers on the earth.
Thus,
extracting the center line of the rail track (CRT) accurately and efficiently, 
to support engineering design, monitor construction quality, 
and ensure operational safety,
has become one of the basic components in the maintenance of existing railways.

CRT extraction can be achieved by real-time kinematics, LiDAR,
and multiple images.
The real-time kinematic is generally mounted on a railway measurement vehicle and obtains the CRT by moving along the rail track.
In general, 
it has a satisfactory accuracy while requiring operations on the track,
thus demanding the cooperation of railway departments, 
and there are issues related to both safety and efficiency.
LiDAR sensors can be mounted on a drone, 
which is more convenient and secure than real-time kinematic. 
Because a further process, 
like point segmentation or classification,
is required for CRT extraction,
the drone must maintain a low flight altitude to satisfy the standards of the point-cloud density,
which would impact the efficiency.
A drone with cameras can capture multiple images efficiently with a safe distance from the railway area.
But CRT extraction is challenging in multiple images:
(1) The dense points reconstructed with images are inaccurate around the railway track because of the occlusion and matching problems caused by the parallax variation.
(2) Joining image semantics to obtain CRT might be workable;
However,
how to detect the semantics of CRT accurately and completely in multiple images remains to be studied.















\bibliographystyle{plainnat}
\bibliography{cas-refs}


\end{document}

