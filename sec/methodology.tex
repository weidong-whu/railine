\section{Methodology}

The flow of our methodology is presented in figure 1.
We take the aerial images of the railway aera as the input,
for which SfM and point clouds are employed in advanced with existing softwares.
We first extract the image line and convert it to space line with the locally optimal plane of the point cloud.
Then,
we cluster the single 3D line to RT candidates with the frame work derived from DBSCAN,  
during which the contexture information of multiple images extracted from ResNet is used as one of the inlier distance,
and the locally SAM are exploited to reduce the seed of DBSCAN to control the false positive.
Having obtained the RT cluster, we then extract the vector-based RT with RANSAC and refine it with our triangulation methods,
which fully exploit the RT structure to resolve the uncertrainty caused by initial image line segment extraction and the point cloud error.

\subsection{Initial 3D line reconstruction}

The initial 3D line $\mathrm{L}$ is the basic cell for the following method.
For each 2D line $\mathbf l=[\mathbf p_1]$ detected in each image,
the two endpoint 3D candidate line segment must be on the 
we randomly sample three space points around $\mathbf l$ and obtain the space plane $\mathrm P$ that $\mathbf l$ may lie on;
then, 
the 3D candidate line segment of $\mathbf l$ corresponding to $\mathrm P$  must be on the two rays is obtained by the ray-to-plane intersection:
\begin{equation}
    \bar{\mathrm L}\in R^{3\times2}=\textit{Inter} \left(\mathrm P, \mathrm C+\mathrm M^{-1} \left[t_1 \mathbf p_1, t_2 \mathbf p_2 \right] \right) 
\end{equation}



and generate $n$ space planes $\left\{ \mathrm{P}_i\right\}_{i=1}^n$ 
and generate the initial 3D line via projecting it into the local 3D plane ,
where {P} is calculated with sampled points around the endpoints of image line.
Due to the uncertainty of point cloud quality,
it is hard to use a fixed distance threshold to define the inliers










