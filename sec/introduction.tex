\section{Introduction}

Currently,
the length of the railway has exceeded 1.3 million kilometers on the earth.
Thus,
extracting the center line of the rail track (CRT) accurately and efficiently, 
to support engineering design, monitor construction quality, 
and ensure operational safety,
has become one of the basic components in the maintenance of existing railways.

CRT extraction can be achieved by real-time kinematics, LiDAR,
and multiple images.
The real-time kinematic is generally mounted on a railway measurement vehicle and obtains the CRT by moving along the rail track.
In general, 
it has a satisfactory accuracy while requiring operations on the track,
thus demanding the cooperation of railway departments, 
and there are issues related to both safety and efficiency.
LiDAR sensors can be mounted on a drone, 
which is more convenient and secure than real-time kinematic. 
Because a further process, 
like point segmentation or classification,
is required for CRT extraction,
the drone must maintain a low flight altitude to satisfy the standards of the point-cloud density,
which would impact the efficiency.
A drone with cameras can capture multiple images efficiently with a safe distance from the railway area.
But CRT extraction is challenging in multiple images:
(1) The dense points reconstructed with images are inaccurate around the railway track because of the occlusion and matching problems caused by the parallax variation.
(2) Joining image semantics to obtain CRT might be workable;
However,
how to detect the semantics of CRT accurately and completely in multiple images remains to be studied.
I have changed in overleaf












